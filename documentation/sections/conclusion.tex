%
% conclusion.tex
%
% Copyright (C) 2018 by Andre Martins Pio de Mattos <andrempmattos@gmail.com>.
%
% Internship Report
%
% This work is licensed under the Creative Commons Attribution-ShareAlike 4.0
% International License. To view a copy of this license,
% visit http://creativecommons.org/licenses/by-sa/4.0/.
%

%
% \brief Title page.
%
% \author Andre Martins Pio de Mattos <andrempmattos@gmail.com>
%
% \version 0.1.0
%
% \date 08/09/2019
%

\newpage

\section{Conclusion} \label{sec:conclusion}

The development of this work lead to relevant outcomes for the research laboratory, since the payload is currently being used for different radiation assessments in the memory devices and is planned to be launched alongside the GOLDS-UFSC satellite in a complete space mission. Concerning the technologies involved in the payload, the approach and design presented some innovations in the previous similar missions since some new concepts were employed, such as the use of a RTOS for more robust verification firmware systems and the creation of a multipurpose device in relatively controlled budget.

The final verification and integration tests presented satisfactory overall performance and reliability, but revealed some bottlenecks. In firmware, the payload allowed a robust framework for future utilization and good maintenance. In hardware, the board work properly from the first production, but presented limitations in the memory frequency operation (in comparison with the maximum operation allowed). In the FPGA design, the system performed properly, but more investigation to improve the radiation resilience could increase the reliability of the experiments and its duration before presenting critical failures.

In summary, the payload accomplished its requirements and purpose, brought some innovations to the previous employed methods, presented points for improvement, and provided the necessary tools for personal development and learning experience.



\clearpage